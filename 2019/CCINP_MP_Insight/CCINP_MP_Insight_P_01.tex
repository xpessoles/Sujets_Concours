\section{Validation des capacités de positionnement du système de déploiement}

\begin{obj}
Vérifier l’exigence 002 «Position de la pince» afin que le point de préhension $P$ du 
système de déploiement DPL puisse être défini à partir de deux coordonnées articulaires.
\end{obj}

%01
\question{Établir la relation vectorielle entre $X_P$, $Y_P$, $L$ et $\vx{0}$, $\vy{0}$, $\vx{1}$ et $\vx{2}$. }


%02
\question{Projeter la relation précédente selon $\vx{0}$ et $\vy{0}$, puis donner les deux équations scalaires 
correspondantes.}

%03
\question{Exprimer $\theta_1$ et $\theta_2$ en fonction de $X_P$, $Y_P$ et $L$. Conclure quant au respect de «l’exigence 
002».}

