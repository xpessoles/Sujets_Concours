\section{Validation des capacités de positionnement du système de déploiement}

\begin{obj}
Vérifier l’exigence 002 «~Position de la pince~» afin que le point de préhension $P$ du 
système de déploiement DPL puisse être défini à partir de deux coordonnées articulaires.
\end{obj}

%01
\question{Établir la relation vectorielle entre $X_P$, $Y_P$, $L$ et $\vx{0}$, $\vy{0}$, $\vx{1}$ et $\vx{2}$. }
\ifprof
\begin{corrige}%
$\vect{OP}=\vect{OQ}+\vect{QP} \Leftrightarrow X_p(t) \vx{0} + Y_p(t) \vy{0} =L\vx{1}+L\vx{2}$.
\end{corrige}
\else
\fi

%02
\question{Projeter la relation précédente selon $\vx{0}$ et $\vy{0}$, puis donner les deux équations scalaires 
correspondantes.}
\ifprof
\begin{corrige}%
$X_P(t)\vx{0}+Y_P(t)\vy{0}=L\left(\cos(\theta_1)\vx{0}+\sin(\theta_1)\vy{0}\right)+L\left(\cos(\theta_1+\theta_2)\vx{0}+\sin(\theta_1+\theta_2)\vy{0}\right)$

$\Rightarrow 
\left\{
\begin{array}{l}
X_P(t)=L\cos(\theta_1)+L\cos(\theta_1+\theta_2) \\
Y_P(t)=L\sin(\theta_1)+L\sin(\theta_1+\theta_2) 
\end{array}
\right.$


\end{corrige}
\else
\fi


%03
\question{Exprimer $\theta_1$ et $\theta_2$ en fonction de $X_P$, $Y_P$ et $L$. Conclure quant au respect de «l’exigence 002».}
\ifprof
\begin{corrige}%
En utilisant les relations trigonométriques : 
$\left\{
\begin{array}{l}
X_P(t)=2L \cos\left(\dfrac{2\theta_1 + \theta_2}{2}\right) \cos
\left(\dfrac{\theta_2}{2}\right) \\
Y_P(t)=2L \sin\left(\dfrac{2\theta_1 + \theta_2}{2}\right) \cos
\left(\dfrac{\theta_2}{2}\right) 
\end{array}
\right.
$

Il vient : 
$X_P^2 (t)+Y_P^2 (t)=4L^2 \cos^2 \left(\dfrac{\theta_2}{2}\right)$ 
$\Leftrightarrow 
\arccos 
\left( 
\dfrac{X_P^2 (t)+Y_P^2 (t)-2L^2}{2L^2}
\right)$

Et : $\dfrac{Y_P(t)}{X_P(t)} =\tan\left(\dfrac{2\theta_1 + \theta_2}{2}\right)$ 
$ \Rightarrow \theta_1 = \arctan \dfrac{Y_P(t)}{X_P(t)} -\dfrac{1}{2} \arccos\left(\dfrac{X_P^2 (t)+Y_P^2 (t)-2L^2)}{2L^2}\right)$

Conformément à l'exigence 002 , la position de la pince peut être commandée par deux variables angulaires.
\end{corrige}
\else
\fi

