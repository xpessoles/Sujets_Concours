
\subsection*{Validation de la capacité statique du système de déploiement}

\begin{obj}
Déterminer le couple statique du moto-réducteur $\indice{M}{01}$ qui permet l’équilibre du système de déploiement.
\end{obj}
1
On note $\vect{g} = - g \vy{0}$ l'accélération du champ de pesanteur terrestre avec $g = \SI{9,81}{ms^{-2}}$.

\question{Exprimer puis calculer le couple statique, noté $\indice{C}{01}$, que doit exercer le moto-réducteur $\indice{M}{01}$ dans
la position du système de déploiement la plus défavorable. Préciser clairement le système isolé ainsi que le principe/théorème utilisé.}

\question{En déduire la valeur minimale du couple de maintien, noté $\indice{C}{m1min}$, dont doit disposer le moteur pas à pas.}
