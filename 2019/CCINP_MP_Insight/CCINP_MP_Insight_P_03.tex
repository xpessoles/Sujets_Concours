\section{Validation de la capacité statique du système de déploiement}

\begin{obj}
Déterminer le couple statique du moto-réducteur $\indice{M}{01}$ qui permet l’équilibre du système de déploiement.
\end{obj}

On note $\vect{g} = - g \vy{0}$ l'accélération du champ de pesanteur terrestre avec $g = \SI{9,81}{ms^{-2}}$.

% Q06
\question{Exprimer puis calculer le couple statique, noté $\indice{C}{01}$, que doit exercer le moto-réducteur $\indice{M}{01}$ dans la position du système de déploiement la plus défavorable. Préciser clairement le système isolé ainsi que le principe/théorème utilisé.}
\ifprof
\begin{corrige}%
On isole (1,2,3).

Bilan des efforts : 
\begin{itemize}
\item Pesanteur sur  1 
\item Pesanteur sur 2
\item Pesanteur sur 3
\item Couple moteur sur 1
\item Liaison pivot. 

\end{itemize}

La position la plus défavorable est $\theta_1=\theta_2=0$.
On applique le TMS en $O$ en projection sur $\vz{}$.

$C_{01}=\dfrac{L}{2} m_1g+3\dfrac{L}{2} m_2g+2Lm_Sg \Leftrightarrow C_{01}=L\left(\dfrac{m_1}{2}+3\dfrac{m_2}{2}+2mS\right)g$

Numériquement :
$C_{01}=0,5\left(\dfrac{0,352}{2}+\dfrac{3\times 0,352}{2}+2.1,2\right) \times 9,81=\SI{15,2}{N.m}$.

\end{corrige}
\else
\fi


% Q07
\question{En déduire la valeur minimale du couple de maintien, noté $\indice{C}{m1min}$, dont doit disposer le moteur pas à pas.}
\ifprof
\begin{corrige}%
$\indice{C}{m1 min}= 15,2/82 = \SI{0,18}{N.m}$.
\end{corrige}
\else
\fi
