\section{Validation du non-dépassement de la vitesse de la sphère SEIS}

\begin{obj}
Valider l’exigence 003 << Vitesse de la pince >> quand la sphère SEIS se déplace en 
translation afin de conserver toujours la même orientation : la vitesse de la pince ne doit pas excéder $\SI{20}{mm/s}$.
\end{obj}

On note $\vectv{M}{S}{R}$ est le vecteur vitesse du point $M$ appartenant au solide $S$ par rapport à $R$.


%Q04
\question{Déterminer l’expression de la vitesse du point $P$, appartenant à l’avant-bras 2, par rapport à $\rep{0}$ en fonction de $\theta_1$, $\theta_2$ et $L$.}

%
\question{Déterminer l’expression de l'accélération du point $P$, appartenant à l’avant-bras 2, par rapport à $\rep{0}$.}

%Q05
\question{Déterminer la valeur maximale du taux de rotation $|| \vecto{1}{0} ||$  pour que l’avant-bras 2 suive un mouvement de translation circulaire par rapport à $\rep{0}$ en respectant l’exigence 003 << Vitesse de la pince >>.}
