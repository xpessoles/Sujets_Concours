%%%%%%%%%%%%%%%%%%%%%%%%%%%%%%%%%%%%%%%%%%%%%%%%
% Corrigé UPSTI
% Concours - Epreuve - Année
%%%%%%%%%%%%%%%%%%%%%%%%%%%%%%%%%%%%%%%%%%%%%%%%
% !TeX encoding = utf8
% !TeX spellcheck = fr

\documentclass[11pt]{article}

%%%%%%%%%%%%%%%%%%%%%%%%%%%%%%%%%%%%%%%%%%%%%%%%
% Package UPSTI_Document
%%%%%%%%%%%%%%%%%%%%%%%%%%%%%%%%%%%%%%%%%%%%%%%% 
\usepackage{UPSTI_Corrige_Concours}	% Squelette minimal
%\usepackage[UPSTI]{UPSTI_Corrige_Concours} % Chargement des packages UPSTI  (téléchargeables ici: https://www.upsti.fr/documents-pedagogiques/upsti-kit-de-demarrage-latex)

%---------------------------------%
% Packages personnalisés
%---------------------------------%
% Insérez ici les packages que vous utilisez habituellement





% ---

%---------------------------------%
% Paramètres du corrigé
%---------------------------------%

% ----------
% Concours
% ----------
% 0: Custom*
% 1: ATS
% 2: Banque PT
% 3: CCINP
% 4: CCP
% 5: CCS (par défaut)
% 6: E3A
% 7: ICNA
% 8: Mines AADN
% 9: Mines Ponts
% 10: X-ENS
% * Si on met la valeur 0, il faut décommenter la ligne suivante: 		
%\newcommand{\UPSTIconcoursCustom}{Concours custom}
\newcommand{\UPSTIidConcours}{2}

% ----------
% Filière
% ----------
% 0: Custom*
% 1: ATS
% 2: MP
% 3: MPI
% 4: PSI (par défaut)
% 5: PT
% 6: TSI
% 7: MP2I
% 8: MPSI
% 9: PCSI
% 10: PTSI
% * Si on met la valeur 0, il faut décommenter la ligne suivante: 		
%\newcommand{\UPSTIfiliereCustom}{Filière custom}
\newcommand{\UPSTIidFiliere}{10}

% ----------
% Epreuve
% ----------
% 0: Custom*
% 1: S2I (par défaut)
% 2: Informatique
% 3: Modélisation et informatique
% 4: Modélisation
% 5: Physique - SI
% 6: SI A
% 7: SI B
% 8: SI C
% 9: SI 1
% 10: SI 2
% * Si on met la valeur 0, il faut décommenter la ligne suivante: 		
%\newcommand{\UPSTIepreuveCustom}{Epreuve custom}
\newcommand{\UPSTIidEpreuve}{7}

% ----------
% Session
% ----------
\newcommand{\UPSTIsession}{2022}

% ----------
% Titre de l'épreuve (souvent, le nom du support)
% ----------
\newcommand{\UPSTItitreEpreuve}{Banc d'essai de lames de tondeuse à gazon}
% Si le nom est trop long pour l'entête, on peu décommenter la ligne suivante:
%\newcommand{\UPSTItitreEpreuveRaccourci}{Titre raccourci}      

%----------------------------------------------- 
\UPSTIprepareDocument		% "Compile" les variables
%%%%%%%%%%%%%%%%%%%%%%%%%%%%%%%%%%%%%%%%%%%%%%%% 


\usepackage{siunitx}
%%%%%%%%%%%%%%%%%%%%%%%%%%%%%%%%%%%%%%%%%%%%%%%% 
% Début du document
%%%%%%%%%%%%%%%%%%%%%%%%%%%%%%%%%%%%%%%%%%%%%%%% 
\begin{document}
\UPSTIpreambuleEpreuve	% Affichage du préambule de l'épreuve

%---------------------------------%
% DEBUT du contenu
%---------------------------------%

\UPSTItitrePartieCorrige{Détermination du couple transmissible par l'arbre}

\section{Détermination du couple transmis par la lame sur l'arbre au moment de l'impact pieu/lame}

\subsection{Couple transmis par adhérence}

\UPSTIquestion* Sur le document réponse, cadre << question 1 >>, identifier par un trait de couleur bleue la surface de contact annulaire entre la lame et la rondelle fusible. 

\begin{UPSTIcorrige}

** image ** 
\end{UPSTIcorrige}

\UPSTIquestion Donner l'expression de $C_{{adh}}$ en fonction de $F_{{vis}}$, $\mu_{{acier/acier}}$, $D_{{int}}$ et $D_{{ext}}$.

\begin{UPSTIcorrige}
Le couple tranmissible par adhérence lorsque la zone de contact est une seule couronne est donné par 
$C_{adh}=\frac{2}{3} \mu_{{acier/acier}}F_{{vis}} \frac{R^3-r^3}{R^2 - r^2}$.
 ou encore  $C_{adh}=\frac{2}{3} \mu_{{acier/acier}}F_{{vis}} \frac{2^2}{2^3}\frac{D_{{ext}}^3-D_{{int}}^3}{D_{{ext}}^2 - D_{{int}}^2}=\frac{1}{3} \mu_{{acier/acier}}F_{{vis}} \frac{D_{{ext}}^3-D_{{int}}^3}{D_{{ext}}^2 - D_{{int}}^2}$.
\end{UPSTIcorrige}


\UPSTIquestion Estimer la valeur de $C_{{adh}}$ (en \si{N.m}) à $\pm\SI{10}{\%}$.

\begin{UPSTIcorrige}
$C_{adh}=\frac{0,15}{3} \times 30000 \times \frac{60^3-32,5^3}{60^2 - 32,5^2}$
$\simeq 1500\times \frac{216000-27000}{3600 - 900}$
$\simeq 1500\times \frac{2000}{30}$
$\simeq 50\times 2000$
$\simeq \SI{100 000}{Nmm}$
soit $C_{adh}\simeq \SI{100}{Nm}$.

\textit{AN : $C_{adh}\simeq \SI{107}{Nm}$.}

\end{UPSTIcorrige}

\subsection{Couple transmis par les ergots}

\UPSTIquestion* Sur le document réponse, cadre << question  4>>, identifier par un trait de couleur rouge la trace de la section cisaillée d'un ergot lors d'un choc sur la lame.

\begin{UPSTIcorrige}
** image ** 
\end{UPSTIcorrige}

\UPSTIquestion Donner l'expression de $C_{cis}$ en fonction des données nécessaires. 

\begin{UPSTIcorrige}
La contrainte de cisaillement peut s'exprimer par $\sigma_c = \frac{F_c}{eb}$.
D'une part, il y a rupture de l'ergot lorsque $\sigma_c \geq R_m$. Au moment de la rupture, on fait donc l'hypothèse que
$R_m = \frac{F_c}{eb}$
D'autre part, pour un seul ergot, $C_{cis1} = F_c \frac{d_{ergot}}{2}$.
On a donc $C_{cis1} = R_m e b \frac{d_{ergot}}{2}$. 

Pour 4 ergots, $C_{cis} = 2R_m e b d_{ergot}$. 
\end{UPSTIcorrige}

\UPSTIquestion Estimer la valeur de $C_{cis}$ à  $\pm\SI{10}{\%}$.

\begin{UPSTIcorrige}
On a dans ce cas, $C_{cis} = 2 \times 300 \times 1,5 \times  4 \times 42$
$ \simeq 2 \times 300 \times 252$ $ \simeq \SI{150000}{Nmm}$ $\simeq \SI{150}{Nm}$ .

\end{UPSTIcorrige}

\subsection{Couple transmis à l'arbre par la rondelle fusible lors d'un chox sur la lame}
\UPSTIquestion* Parmi les cinq expressions proposées dans le document réponse, cocher l'expression correcte de $C_{rond}$ en fonction de 
$C_{adh}$ et $C_{cis}$.

\begin{UPSTIcorrige}
Lors d'un choc, on veut que les ergots soient cisaillés. Pour cela il faut donc qu'il y ait glissement entre la rondelle et la lame puis que le couple généré par le chox cisaille les ergots. 
Nécessairement, on a donc $C_{rond}=MAX\left(C_{adh},C_{cis}\right)$.
\end{UPSTIcorrige}

\UPSTIquestion Donner la valeur de $C_{rond}$.

\begin{UPSTIcorrige}
En conséquence, $C_{rond} = MAX(120,200) = \SI{200}{Nm}$. 
\end{UPSTIcorrige}

\section{Détermination du couple transmis par le moteur électrique sur l'arbre au moment du démarrage}

\UPSTIquestion* Afin de déterminer $C_{mot}$, préciser l'ensemble isolé, cocher le principe retenu, le théorème utilisé et l'axe sur lequel il sera projeté (voir fig. 15, annexe B). 

\begin{UPSTIcorrige}
En utilisanat l'hypothèse que l’intégralité du couple est transmise à l'arbre, on choisit d'isoler l'arbre. 

Pour déterminer le couple moteur, on choisit << naturellement >> d'isoler l'arbre, d'appliquer le Prinicipe Fondamental de la Dynamique et plus précisément le théorème du moment dynamique en $A$ en projection sur l'axe de rotation à savoir sur l'axe $\left(A,\overrightarrow{x_a}\right)$.
 \end{UPSTIcorrige}

\UPSTIquestion Donner l'expression de $C_{mot}$ en fonction des données de $J_{lame}$, $N_{arbre}$ et $\Delta t_{acc}$. 
\begin{UPSTIcorrige}
\end{UPSTIcorrige}

\UPSTIquestion Donner la référence de la lame (vour annexe C) qui est la plus exigeante pour dimensionner le couple moteur. 
\begin{UPSTIcorrige}
\end{UPSTIcorrige}


\UPSTIquestion Calculer le couple minimal $C_{mot\_min}$ que doit exercer le moteur sur l'arbre pour respecter $\Delta t _{acc}  = \SI{10}{s}$ quelle que soit la lame testée. 
\begin{UPSTIcorrige}
\end{UPSTIcorrige}


\section{Détermination du couple maximal transmssible par l'arbre}

\UPSTIquestion* Parmi les expressions proposées dans le document réponse, cocher l'expression correcte de $C_{arbre\_max}$ en fonction de $C_{rond}$ et $C_{mot\_max}$.
\begin{UPSTIcorrige}
\end{UPSTIcorrige}

\UPSTIquestion Donner la valeur de $C_{arbre\_max}$.
\begin{UPSTIcorrige}
\end{UPSTIcorrige}


\UPSTItitrePartieCorrige{Prédimensionnement de l'arbre}
\section{Etude de l'arbre en flexion exclusivement}

\UPSTIquestion* Exprimer les composantes des forces des paliers B et C sur l'arbre en fonction des données.
\begin{UPSTIcorrige}
\end{UPSTIcorrige}

\UPSTIquestion Calculer les valeurs de composantes des forces des paliers B et C sur l'arbre.
\begin{UPSTIcorrige}
\end{UPSTIcorrige}

\UPSTIquestion Donner l'expression des moments de flexion $M_{fzAB}(x_a)$ et $M_{fzBC}(x_a)$ en fonction des données $B_x$, $B_y$, $C_x$ ou $C_z$.
\begin{UPSTIcorrige}
\end{UPSTIcorrige}

\UPSTIquestion Calculer les valeurs des moments de flexion en B et C.
\begin{UPSTIcorrige}
\end{UPSTIcorrige}

\UPSTIquestion Tracer l'évolution du moment de flexion $M_{fz}(x_a)$ dans l'abre en précisant les caleurs significatives. 
\begin{UPSTIcorrige}
\end{UPSTIcorrige}

\UPSTIquestion Identifier le point où le moment de flexion est maximal et donner l'expression de $M_{fz-maxi}$.
\begin{UPSTIcorrige}
\end{UPSTIcorrige}

\UPSTIquestion Donner l'expression de la contrainte normale maximale dans une section droite $\sigma_{max\_sect}$ en fonction de $M_{fz}(x_a)$ et de $d_{flexion}$.
\begin{UPSTIcorrige}
\end{UPSTIcorrige}

\UPSTIquestion Exprimer le diamètre minimum de l'arbre de flexion $d_{flexion\_mini}$ en fonction de $s$, $R_e$ et des données identifiées précédemment. 
\begin{UPSTIcorrige}
\end{UPSTIcorrige}

\UPSTIquestion Calculer le dimaètre minimum de l'arbre en flexion $d_{flexion\_mini}$. On pourra s'appuyer sur la figure 21 de l'annexe E. 
\begin{UPSTIcorrige}
\end{UPSTIcorrige}


\section{Etude de l'arbre en torsion exclusivement}
\UPSTIquestion* En isolant l'ensemble \{arbre, poulie réceptrice\}, écrire le théorème du moment dynamique en proection sur l'axe $\overrightarrow{x_a}$.
\begin{UPSTIcorrige}
\end{UPSTIcorrige}

\UPSTIquestion En isolant uniquement l'arbre, écrire le théorème du moment dynamique en proection sur l'axe $\overrightarrow{x_a}$.
\begin{UPSTIcorrige}
\end{UPSTIcorrige}

\UPSTIquestion En considérant que le moment d'inertie de l'arbre est négligeable devant les autres moments d'inertie, écrire la relation liant $C_{rond\rightarrow arbre}$ et $C_{poulie\rightarrow arbre}$.
\begin{UPSTIcorrige}
\end{UPSTIcorrige}

\section{Analyse des résultats}

\UPSTIquestion* Quelle proposition du document réponse retenez-vous pour déterminer le diamètre minimum de l'arbre ?
\begin{UPSTIcorrige}
\end{UPSTIcorrige}

\UPSTIquestion Les dimensions du châssis et de l'abre ne sont pas figés à ce stade de l'étude. Il est encore possible de faire évoluer les longueurs $l_1$, $l_2$ et $l_3$. A-t-on intérêt à les augmenter ou les diminuer ? Compléter le document réponse en justifiant les choix effectués. 
\begin{UPSTIcorrige}
\end{UPSTIcorrige}


\UPSTItitrePartieCorrige{Etude du ressort d'éjection du pieu}
\section{Détermination de l'énergie nécessaire à l'éjection du pieu}

\UPSTIquestion* Donner l'expression du la massu du pieu en fonction de son diamètre $D_{pieu}$, de sa longueur $L_{pieu}$ et de la masse volumique de l'acier $\rho_{acier}$. Précisez l'unité. 
\begin{UPSTIcorrige}
\end{UPSTIcorrige}

\UPSTIquestion Donner une valeur numérique pour la masse volumique (deux chiffres significatifs) d'un acier standard $\rho_{acier}$. Précisez l'unité. 
\begin{UPSTIcorrige}
\end{UPSTIcorrige}

\UPSTIquestion Calculer la masse du pieu $m_{pieu}$. Précisez l'unité. 
\begin{UPSTIcorrige}
\end{UPSTIcorrige}


\UPSTIquestion Compte tenu des indications précédentes, donner la valeur numérique de la course $\Delta L_{finale}$. 
\begin{UPSTIcorrige}
\end{UPSTIcorrige}

\UPSTIquestion Ecrire la relation littérale puis calculer $t_{\Delta finale}$, en secondes.
\begin{UPSTIcorrige}
\end{UPSTIcorrige}

\UPSTIquestion Ecrire la relation littérale puis calculer la vitesse du pieu $V_{finale}$, en $\si{m.s^{-1}}$.
\begin{UPSTIcorrige}
\end{UPSTIcorrige}

\UPSTIquestion Ecrire la relation littérale puis calculer l'énergie cinétique du pieu $E_{finale tir}$. 
\begin{UPSTIcorrige}
\end{UPSTIcorrige}

\section{Dimensionnement du ressort propulseur}
\subsection*{Choix n°1}
\UPSTIquestion* Sur la figure 23 de l'annexe G, que représente l'aire grisée ?
En déduire la course du tir $C_{poussée choix 1}$ correspondant à ce choix 1. Ecrire la relation littérale puis effectuer le calcul numérique. 
\begin{UPSTIcorrige}
\end{UPSTIcorrige}


\subsection*{Choix n°2}
\UPSTIquestion* Sur le document réponse, compléter la figure en ajoutant les éléments suivants : 
\begin{itemize}
\item évolution de la force développée par le ressort;
\item $F_{fin\, poussee\, choix 2}$;
\item le travail du ressort pendant la poussée.
\end{itemize}
\begin{UPSTIcorrige}
\end{UPSTIcorrige}

\UPSTIquestion De la figure précédente, déduire l'expression littérale de $F_{fin pouss\'ee choix 2}$ en fonction de $E_{ressort}$, $F_{op max choix 2}$ et $C_{pouss\'ee choix 2}$ puis effectuer le calcul numérique. 

\begin{UPSTIcorrige}
\end{UPSTIcorrige}



\subsection*{Utilisation d'un configurateur en ligne}

\UPSTIquestion* En tenant compte des données et des résultats de vos calculs (choix 2), sur le document réponse, compléter uniquement les cases nécessaires de la fiche de calcul du constructeur.
\begin{UPSTIcorrige}
\end{UPSTIcorrige}

\UPSTIquestion Donner l'expression puis calculer les efforts $F_{armée}$ du ressport proposé en position << armé >> et l'effort $F_{fin de tir}$ en position << fin de tir >>. 

Donner le diamètre extérieur nominal $D_{ext nominal}$ et calculer le diamètre extérieur $D_{ext max}$. Conclure. 
\begin{UPSTIcorrige}
\end{UPSTIcorrige}

\subsection*{Identification des zones de contact}
\UPSTIquestion* Sur le document réponse, compléter la cotation de la bague << glycodur >> et du pêne. Calculer le jeu $J_{max}$ en mm.
\begin{UPSTIcorrige}
\end{UPSTIcorrige}

\UPSTIquestion Exprimer de façon littérale puis estimer en radian la valeur de l'angle de rotulage $\alpha_{max}$.
Préciez l'hypothèse utilisée pour réaliser votre estimation d'angle. 
\begin{UPSTIcorrige}
\end{UPSTIcorrige}

\subsection*{Détermination de l'effort à fournir par l'actionneur électromagnétique.}
\UPSTIquestion* Sur le document réponse, en position << tir >>, à l'échelle : 
\begin{itemize}
\item en rouge, tracer $P_N$, $Q_N$ et $R_N$;
\item en bleu, tracer $P_T$, $Q_T$ et $R_T$;
\item en bleu, tracer $\overrightarrow{F_{sol tir}}$.
\end{itemize}
Donner la valeur numérique de $P_T$, $Q_T$ et $R_T$.
\begin{UPSTIcorrige}
\end{UPSTIcorrige}

\UPSTIquestion Ecrire l'expression littérale de la norme de $\overrightarrow{F_{sol tir}}$ et donner sa valeur numérique.
\begin{UPSTIcorrige}
\end{UPSTIcorrige}

\UPSTIquestion Sur le document réponse, cocher la case correspondant au taux d'utilisation des actionneurs électromagnétiques du banc d'essai.
\begin{UPSTIcorrige}
\end{UPSTIcorrige}

\UPSTIquestion Parmi les trois actionneurs proposés, choisir celui ou ceux qui conviennent en entourant sur chaque graphique la zone de la courbe qui justifie ce choix.
\begin{UPSTIcorrige}
\end{UPSTIcorrige}


\UPSTItitrePartieCorrige{Dessin d'étute de Construction Mécanique}

\section{Platine de fixation 16}

\UPSTIquestion* Sur le document pré imprimé format A3, cadre << Mise en situation >> à l'échelle 1/2, choisir l'orientation du profilé donnant la meilleure rigidité du châssis en cochant la case S1 ou S2. 
\begin{UPSTIcorrige}
\end{UPSTIcorrige}

\UPSTIquestion Représenter votre proposition de solution pour la liaison complète entre la platine de fixation 16 et le bâti. Le maintien en position est déjà défini (6 vis et 6 écrous spéciaux pour profilé) Lorsque les vis ne sont pas serrées, le réglage de la position du canon par rapport au bâti doit être possible.

Indiquer la valeur des ajustements normalisés nécessaires.
\begin{UPSTIcorrige}
\end{UPSTIcorrige}

\UPSTIquestion Représenter votre proposition de solution pour la liaison complète (appui-plan + centrage court) démontable entre le fût 17 et la platine de fixation 16. 

Indiquer la valeur des ajustements normalisés nécessaires.
\begin{UPSTIcorrige}
\end{UPSTIcorrige}

\UPSTIquestion Représenter votre proposition de solution pour la liaison complète (appui-plan + centrage court) démontable entre le corps du propulseur 1 et la platine de fixation 16. Indiquer la valeur des ajustements normalisés nécessaires.

Cette liaison doit peermettre également la mise et le maintien en position de la bague d'amortissement 15 entre la platine de fixation 16 et le corps du propulseur 1. 

\begin{UPSTIcorrige}
\end{UPSTIcorrige}

\section{Gâchette}

\UPSTIquestion* Représenter votre proposition de solution pour la liaison entre le pêne 10 et l'axe du solénoïde 13. Cette liaison ne doit permettre que la transmission de l'effort de dévérouillage. 

Les jeux permettant les degrés de liberté nécessaires au bon fonctionnement doivent être représentés (1mm minimum). 
\begin{UPSTIcorrige}
\end{UPSTIcorrige}

\UPSTIquestion Représenter votre proposition de solution pour la forme extérieure du poussoir 7 ainsi que la forme de l'extrémité du pêne 10, permettant, lors de l'aménagement, de repousser l'axe du solénoïde pour rélaiser le vérouillage automatique du système. En position << armée >>, le contact linéique entre le pêne 10 et la bague << Glycodur >> 8 doit être de 2 mm au minimum.
\begin{UPSTIcorrige}
\end{UPSTIcorrige}

\section{Culasse 2 et poussoir 7}

\UPSTIquestion* La liaison complète entre la culasse 2 et le corps du propulseur 1 est partiellement réalisée. Le maintien en position par 3 vis est déjà défini. Représenter votre proposition de solution permettant la mise en position de la culasse 2 sur le corps du propulseur 1. 
\begin{UPSTIcorrige}
\end{UPSTIcorrige}

\UPSTIquestion Définir les formes de la clulasse 2 permettant de guider le ressort propulseur3.
\begin{UPSTIcorrige}
\end{UPSTIcorrige}

\UPSTIquestion Représenter votre proposition de solution permettant la mise et le maintien en position des douilles à billes dans la culasse 2. 
\begin{UPSTIcorrige}
\end{UPSTIcorrige}

\UPSTIquestion Définir les formes du poussoir 7 permettant de guider le ressort propulseur 3.

 Représenter votre proposition de solution pour la liaison complète démontable entre le poussoir 7 et la tige d'armement 5. 
\begin{UPSTIcorrige}
\end{UPSTIcorrige}


% -------------------------- 
% Boite d'objectif 
% -------------------------- 
%\UPSTIobjectif{
%Exemple d'objectif de partie...
%}
% -------------------------- 


\end{document}
