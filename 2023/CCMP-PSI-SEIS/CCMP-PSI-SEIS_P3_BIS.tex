\section{Caractérisation dynamique de l'ensemble mobile}
On rappelle que les pendules inversés du sismomètre VBB conservent leur équilibre autour de leur axe de rotation $\left(O_{1}, \overrightarrow{z_{1}}\right)$ par rapport au bâti grâce à un ressort à lame souple. Afin que chaque système soit suffisamment sensible aux séismes sur Mars, le choix du ressort associé à une articulation à lamelles doit également permettre d'amplifier les mouvements sur la plage de fréquences attendues pour les séismes martiens.

On définit pour cela les exigences de la table \ref{ccmp2023_tab_02}.


\begin{table}[!h]
\centering
\begin{tabular}{lp{5cm}p{5cm}p{4cm}}
\hline
$\mathbf{2}$ &\multicolumn{3}{l}{Être mécaniquement sensible aux séismes attendus sur Mars} \\
\hline
$\mathbf{2.1}$ & Être suffisamment sensible & Amplification mécanique & $>2 \mathrm{rad} \cdot \mathrm{m}^{-1} \cdot \mathrm{s}^{2}$ \\
\hline
$\mathbf{2.2}$ & Être sensible aux fréquences des séismes attendus sur Mars & 
Amplification en fonction de la fréquence des mouvements du sol & $\geq 10 \si{dB}$ dans la bande $[0,01 ; 0,5] \si{Hz}$ soit $[0,06 ; 3] \si{rad} \cdot \mathrm{s}^{-1}$  \\
\hline
\end{tabular}
\caption{Liste (non exhaustive) des exigences de sensibilité mécanique d'un système \label{ccmp2023_tab_02}}
\end{table}

Chaque ressort est unique, et entre les trois systèmes constituant VBB, ces derniers ne sont pas interchangeables. Ils sont fabriqués sur mesure, en tenant compte des caractéristiques des articulations à lamelles, uniques elles aussi.

\begin{obj}
Etablir le lien entre la raideur du ressort et la pulsation propre de l'ensemble mobile; choisir un ressort et une articulation à lamelles de façon à respecter les exigences 2.1 et 2.2.
\end{obj}


\subsection{Modélisation dynamique de l'ensemble mobile en réponse à un séisme}
En cas de séisme, le sol (1) est en mouvement. Il entraîne dans son mouvement le bâti du système et ne peut plus être considéré comme un référentiel galiléen.
%
%Le schéma cinématique et le paramétrage du dispositif sont fournis en Annexe 4, ainsi que l'ensemble des notations et hypothèses utiles pour cette sous-partie.
%
%On admet qu'il y a un mouvement de translation de (1) par rapport au repère $R_{0}$ dans les directions $\overrightarrow{x_{1}}$ et $\overrightarrow{y_{1}}$, comme imposé par les deux liaisons glissières en série entre (1) et $R_{0}$ sur le schéma cinématique de l'Annexe 4. Aucun degré de liberté en rotation n'est admis : $\vec{\Omega}_{1 / 0}=\overrightarrow{0}$.
%
%Un logiciel de CAO a permis de déterminer numériquement la matrice d'inertie du pendule (2) équipé du contrepoids (3) en $O_{1}$, où tous les termes sont en $\mathrm{kg} \cdot \mathrm{m}^{2}$ :
%
%$$
%\overline{{I}}_{\left(O_{1}, 2+3\right)}=\left(\begin{array}{ccc}
%5,31 \times 10^{-4} & -0,94 \times 10^{-4} & -5,67 \times 10^{-7} \\
%-0,94 \times 10^{-4} & 1,25 \times 10^{-4} & -2,33 \times 10^{-8} \\
%-5,67 \times 10^{-7} & -2,33 \times 10^{-8} & 4,04 \times 10^{-4}
%\end{array}\right)_{\left(\overrightarrow{x_{2}}, \overrightarrow{y_{2}}, \overrightarrow{z_{1}}\right)}=\left(\begin{array}{ccc}
%I_{x x} & -I_{x y} & -I_{x z} \\
%-I_{x y} & I_{y y} & -I_{y z} \\
%-I_{x z} & -I_{y z} & I_{z z}
%\end{array}\right)_{\left(\overrightarrow{\left.x_{2}, \overrightarrow{y_{2}}, \overrightarrow{z_{1}}\right)}\right.}
%$$
%
%%Q12. 
%\question{\label{ccmp2023_q_12}Quel plan de symétrie de l'ensemble mobile $\{(2)+(3)\}$ serait cohérent avec les valeurs numériques de la matrice d'inertie $\overline{{I}}_{\left(O_{1}, 2+3\right)}$ déterminée par le logiciel de CAO ? En déduire, sous sa forme littérale, l'écriture simplifiée de la matrice d'inertie.}
%
%\question{\label{ccmp2023_q_13}Déterminer, dans son mouvement par rapport au repère $R_{0}$, l'expression du moment cinétique de $\{(\mathbf{2})+(3)\}$ en $O_{1}, \vec{\sigma}_{O_{1},(2+3) / R_{0}}$. On l'exprimera en fonction des paramètres cinétiques de $\{(2)+(3)\}$ et des paramètres géométriques et cinématiques du système.}
%
%\question{\label{ccmp2023_q_14}Montrer que la projection sur $\overrightarrow{z_{1}}$ du moment dynamique de $\{(2)+(3)\}$ dans son mouvement par rapport au repère $R_{0}$ en $O_{1}$, est de la forme suivante :
%$$
%\vec{\delta}_{O_{1},(2+3) / R_{0}} \cdot \overrightarrow{z_{1}}=I_{z z} \ddot{\alpha}(t)-d M \gamma_{x 2}(t)
%$$
%où l'on précisera l'expression de $\gamma_{x 2}(t)$.}
%
%On déduit de cette équation que $\overrightarrow{x_{2}}$ est la direction de sensibilité de l'ensemble mobile, c'est-à-dire que l'ensemble mobile n'est sensible qu'aux accélérations du sol en projection sur $\overrightarrow{x_{2}}$.

%Q15.
%\question{\label{ccmp2023_q_12}Préciser, sans faire de calculs, le système isolé et l'équation issue du Principe Fondamental de la Dynamique qui permet d'obtenir l'équation du mouvement de l'ensemble mobile suivante :
%\begin{equation*}
L'équation issue du Principe Fondamental de la Dynamique permet d'obtenir l'équation du mouvement de l'ensemble mobile suivante :
$$I_{z z} \ddot{\alpha}(t)+\mu \dot{\alpha}(t)+k\left(\alpha(t)-\alpha_{0}\right)=a M_{2} g_{\mathrm{M}} \sin \alpha(t)+d M \gamma_{x 2}(t)+C_{0}
$$
%\end{equation*}
%Donner les éléments d'application (équation, projection, point éventuel...) du théorème utilisé. Justifier que l'équation obtenue n'est pas linéaire, indépendamment de l'expression de $\gamma_{x 2}(t)$.}

Afin de mettre en évidence les caractéristiques de l'ensemble mobile en réponse à une accélération du sol $\gamma_{x 2}(t)$, ses oscillations ayant une amplitude très faible, l'équation du mouvement est linéarisée autour du point d'équilibre $\alpha_{\mathrm{eq}}=\alpha_{0}$ de l'ensemble mobile.

On pose $\alpha(t)=\alpha_{0}+\Delta \alpha(t)$, avec $\Delta \alpha(t) \ll \alpha_{0}$.

%Q16. 
\question{\label{ccmp2023_q_16}Montrer que l'équation du mouvement linéarisée s'écrit :
$$
I_{z z} \ddot{\Delta} \alpha(t)+\mu \dot{\Delta} \alpha(t)+k \Delta \alpha(t)=a M_{2} g_{\mathrm{M}} \cos \alpha_{0} \Delta \alpha(t)+d M \gamma_{x 2}(t) %\tag{eq.3}
$$}

On note $\alpha(p)$ et $\gamma_{x 2}(p)$ respectivement les transformées de Laplace des variations angulaires $\Delta \alpha(t)$ et de l'accélération du sol $\gamma_{x 2}(t)$. Les conditions initiales sont supposées nulles.

%Q17. 
\question{\label{ccmp2023_q_17}Exprimer, sous forme canonique, la fonction de transfert de l'ensemble mobile $\frac{\alpha(p)}{\gamma_{x 2}(p)}$ et donner la condition de stabilité de l'ensemble mobile sous la forme d'une inéquation. Conclure sur le rôle stabilisateur du ressort.}

%Q18. 
\question{\label{ccmp2023_q_18}Donner, en fonction des constantes du problème, les expressions des constantes caractéristiques de cette fonction de transfert : gain d'amplification mécanique noté $A$, pulsation propre $\omega_{0}$ et coefficient d'amortissement~$\xi$.}


\subsection{Choix du couple ressort/articulation pour le système}
Pour optimiser la conception des systèmes, 50 ressorts à lame souple et 8 articulations à lamelles ont été fabriqués. L'association d'un ressort et d'une articulation confère une certaine raideur $k$ et un certain moment de précontrainte $C_{0}$ sur l'axe de rotation du pendule par rapport au bâti. Grâce au dispositif expérimental mis au point avec le contrepoids, $k$ et $C_{0}$ sont mesurés pour chaque association, ce qui permet de tracer point par point le diagramme de la Figure A du document réponse (question \ref{ccmp2023_q_21}). Pour faciliter l'interprétation du diagramme, seuls 16 points ont été tracés au lieu de 400 .

Le couple idéal ressort/articulation doit être déterminé pour une utilisation du pendule sur Mars, c'est-à-dire sans contrepoids. Ainsi, les équations précédentes $(1,1,2$ et 3 ) restent valables, en substituant $M_{2}$ par $M$ et $a$ par $d$.

De plus, $\alpha_{\mathrm{eq}}=\alpha_{0}$. On admet que $\alpha_{0}=30^{\circ}$ assure un bon compromis entre un gain d'amplification maximisé et un bruit de mesure faible.

Les données numériques utiles sont résumées dans le tableau suivant, dans les Unités du Système International (USI) :

\begin{center}
\begin{tabular}{|c|c|c|}
\hline
$\dd M$ & $g_{M}$ & $\dd M g_{\mathrm{M}} \cos \alpha_{0}$ \\
\hline
$4,4 \times 10^{-3} \si{m} \cdot \mathrm{kg}$ & $4 \si{m} \cdot \mathrm{s}^{-2}$ & $15 \times 10^{-3} \mathrm{USI}$ \\
\hline
\end{tabular}
\end{center}

\question{\label{ccmp2023_q_19}Déterminer la valeur numérique de $C_{0}$ pour assurer l'équilibre du pendule.}

\question{\label{ccmp2023_q_20}Donner les 2 inéquations qui régissent le choix de la raideur $k$ du couple ressort/articulation, afin de satisfaire l'exigence 2.1 et d'avoir un système stable. À l'aide d'applications numériques, en déduire la plage de valeurs acceptables pour $k$ afin de satisfaire ces 2 conditions.}

\question{\label{ccmp2023_q_21}Sur la Figure A du Cahier Réponses, tracer les droites encadrant les valeurs acceptables pour $k$ et la droite correspondant à la valeur idéale de $-C_{0}$. En déduire le meilleur couple ressort/articulation pour le pendule étudié en entourant le point correspondant sur le diagramme.}

Le diagramme de Bode en gain de la fonction de transfert $\frac{\alpha(p)}{\gamma_{x 2}(p)}$ du système pour ce choix du couple ressort/articulation est fourni sur la Figure B du Cahier Réponses (Question \ref{ccmp2023_q_22})

%Q22. 
\question{\label{ccmp2023_q_22}Conclure vis-à-vis de l'exigence 2.2. Les tracés nécessaires devront figurer sur la FIGURE B du Cahier Réponses. Le système en l'état est-il satisfaisant pour la mesure des mouvements du sol martien dans la plage de fréquence des séismes attendus sur Mars?}

