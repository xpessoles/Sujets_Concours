Gestion de versions de grands textes 

Dans ce sujet, on s'intéresse à des textes de grande taille auxquels plusieurs auteurs apportent
des modifications au cours du temps. Ces textes peuvent par exemple être des programmes
informatiques développés par de multiples auteurs. Il est important de pouvoir efficacement
gérer les différentes versions de ces programmes au cours de leur développement et limiter le
stockage et la transmission d'informations redondantes. Nous allons pour cela nous intéresser à
une notion de \textit{différentiels} entre textes. 

\begin{defi}{Complexité}
La complexité, ou le temps d'exécution, d'une fonction $P$ est le nombre d'opérations élémentaires (addition, multiplication, affectation, test, etc ...) nécessaires à l'exécution
de $P$ dans le cas le pire. Lorsque la complexité dépend d'un ou plusieurs paramètres $\kappa_1$, ..., $\kappa_r$, on dit que $A$ a une complexité en $\mathcal{O}\left( f \left(\kappa_1, ..., \kappa_r  \right)\right)$ s'il existe une constante $C > 0$ telle que,
pour toutes les valeurs de $\kappa_1$, ..., $\kappa_r$ suffisamment grandes (c'est-à-dire plus grandes qu'un certain seuil), pour toute instance du problème de paramètres $\kappa_1$, ..., $\kappa_r$, la complexité est au plus
$C \cdot f \left(\kappa_1, ..., \kappa_r  \right)$.
\end{defi}

Lorsqu'il est demandé de donner la complexité d'un programme, le candidat devra justifier
cette dernière si elle ne se déduit pas directement de la lecture du code. 

\textbf{Rappels concernant le langage}  \texttt{Python}
Ce sujet utilise les types Python listes et dictionnaires, mais seules les opérations mentionnées ci-dessous sont autorisées dans vos réponses. Quand une complexité est indiquée avec un symbole (*), cela signifie que nous faisons une hypothèse
simplificatrice sur sa complexité. La justification de cette simplification est hors-programme. 

Si \lstinline{l}, \lstinline{l1}, \lstinline{l2} désignent des listes en \lstinline{Python} : 
\begin{itemize}
\item  \lstinline{len(l)} renvoie la longueur de la liste \lstinline{l}, c'est-à-dire le nombre d'éléments qu'elle contient.
Complexité en $\mathcal{O}(1)$;
\item  \lstinline{l1 == l2} teste l'égalité des listes \lstinline{l1} et \lstinline{l2}. Complexité en $\mathcal{O}(n)$ avec $n$ le minimum de \lstinline{len(l1)} et \lstinline{len(l2)};
\item  \lstinline{l[i]} désigne le \lstinline{i}-ème élément de la liste \lstinline{l}, où l'indice \lstinline{i} est compris entre \lstinline{0} et \lstinline{len(l)-1}. Complexité en $\mathcal{O}(1)$.
\item  \lstinline{l[i: j]} construit la sous-liste \lstinline{[l[i] , ... , l[j-1]]}. Complexité en $\mathcal{O}(j-i)$. L'usage d~
variantes \lstinline{l[i:]} à la place de \lstinline{l[i:len(l)]}, et de \lstinline{l[:j]} à la place de \lstinline{l[0:j]} est aussi
autorisé. 
\item \lstinline{l.append(e)} modifie la liste \lstinline{l} en lui ajoutant l'élément e en dernière position. Complexité
en $\mathcal{O}(1)$ (*).
\item \lstinline{l.pop()} renvoie le dernier élément de la liste \lstinline{l} (supposée non vide) et supprime l'occurrence
de cet élément en dernière position dans la liste. Complexité en $\mathcal{O}(1)$ (*).
\end{itemize} 

On pourra aussi utiliser la fonction range pour réaliser des itérations.
Si d est un dictionnaire Python :
\begin{itemize}
\item \{\lstinline{key_1: v_1, ..., key_n: v_n}\} crée un nouveau dictionnaire en associant chaque valeur
\lstinline{v_i} à une clé \lstinline{key_i}. Complexité en $\mathcal{O}(n)$ (*).
\item \lstinline{d[key]} renvoie la valeur associée à la clé \lstinline{key} dans \lstinline{d} et lève une erreur si la clé key n`est
pas présente. Complexité en $\mathcal{O}(1)$ (*).
\item \lstinline{d[key] = v} modifie \lstinline{d} pour associer la valeur \lstinline{v} à la clé \lstinline{key}, même si la clé \lstinline{key} n`est pas
présente dans \lstinline{d} initialement. Complexité en $\mathcal{O}(1)$ (*).
\item \lstinline{key} in \lstinline{d} teste si la clé \lstinline{key} est présente dans \lstinline{d}. Complexité en $\mathcal{O}(1)$ (*).
\end{itemize}

Sauf mention contraire, les fonctions à écrire ne doivent pas modifier leurs entrées.



La structure de données texte. Dans ce sujet, on appelle texte une liste de caractères. Par
exemple, \lstinline{[`b`, `i`, `n`, `g`, `o`]} est un texte de longueur 5.


\section*{Partie I : Différentiels par positions fixes}
Dans cette partie, nous traitons le problème avec une hypothèse simplificatrice : les textes
comparés ont toujours la même taille.

%Question 1. 
\question{Sans utiliser le test \lstinline{==} sur les listes, écrire une fonction 
\lstinline{textes_égaux(texte1,texte2)} qui teste si deux textes sont égaux. Donner la complexité de cette fonction.}

Exemples
\begin{lstlisting}
>>> textes_égaux([`v`, `i`, `s`, `a`], [`v`, `a`, `i`, `s`])
False
>>> textes_égaux([`v`, `i`, `s`, `a`], [`v`, `i`, `s`, `a`])
True
\end{lstlisting}

Dans la suite de ce sujet, on pourra utiliser \lstinline{==} sur les listes plutôt que cette fonction.


Si deux textes ne sont pas égaux mais ont la même longueur n, on souhaite compter le nombre
de positions qui diffèrent, c`est à dire déterminer combien il existe de positions $i$ ($0\leq i < n$)
telles que les caractères en position $i$ sont différents dans les deux textes.

\question{Écrire une fonction \lstinline{distance(texte1, texte2)} qui calcule cette quantité. On
supposera que les deux textes ont le même nombre de caractères. Donner la complexité de cette
fonction.}

Exemples
\begin{lstlisting}
>>> distance([`v`, `i`, `s`, `a`], [`v`, `a`, `i`, `s`])
3
>>> distance([`a`, `v`, `i`, `s`], [`v`, `i`, `s`, `a`])
4
\end{lstlisting}


%Question 3. 
\question{En vous aidant d`un dictionnaire dont les clés sont des caractères, écrire une fonction \lstinline{aucun_caractère_commun(texte1, texte2)} qui renvoie \lstinline{True} si et seulement si l`ensemble
des caractères qui apparaissent dans \lstinline{texte1} est disjoint de l`ensemble des caractères qui apparaissent dans \lstinline{texte2}. Les deux textes peuvent avoir ici des longueurs différentes. Cette fonction devra avoir une complexité $\mathcal{O}\left(\text{len(texte1)} + \text{len(texte2)}\right)$.}

Exemples
\begin{lstlisting}
>>> aucun_caractère_commun([`a`, `v`, `i`, `s`], [`v`, `i`, `s`, `a`])
False
>>> aucun_caractère_commun([`a`, `v`, `i`, `s`], [`u`, `r`, `n`, `e`])
True
\end{lstlisting}

Nous introduisons maintenant une structure de données spécifique pour représenter un différentiel par positions fixes entre deux textes.
La Figure 1 présente un exemple de couple de textes (texte1, texte2) qui diffèrent sur 4
tranches (représentées par des zones grisées sur la figure). En dehors des tranches, les textes sont
égaux.

\begin{figure}[H]
\caption{Exemple de couple (texte1, texte2) dont on veut calculer le différentiel (sur des
positions fixes)}
\end{figure}

\textbf{La structure de données tranche.} Une tranche est un dictionnaire avec trois clés \lstinline{`début`},
\lstinline{`avant`} et \lstinline{`après`}. La valeur associée à la clé \lstinline{`début`} est le premier indice de la tranche, les
extes associés aux clés \lstinline{`avant`} et \lstinline{`après`} représentent les textes (de même longueur) de la
tranche avant et après modification. Dans la suite de cette partie, on s`appuiera sur les fonctions
suivantes pour manipuler cette structure.


\begin{lstlisting}
def tranche(arg_début, arg_avant, arg_après):
    return {`début`: arg_début, `avant`: arg_avant, `après`: arg_après}

def début(tr):
    return tr[`début`]

def après(tr):
    return tr[`après`]

def avant(tr):
    return tr[`avant`]

def fin(tr):
    return début(tr) + len(après(tr))
\end{lstlisting}

Nous ne fournissons pas de fonction pour modifier une tranche car nous souhaitons traiter
cette structure de données comme une structure immuable\footnote{On rappelle qu’une structure immuable est une structure qui n’est jamais modifiée. C’est par exemple le
cas des chaînes et des tuples en Python.}.
On peut représenter le différentiel de la Figure 1, par la liste suivante :

\begin{lstlisting}
[
tranche(3, [`g`, `r`, `a`, `n`, `d`], [`p`, `e`, `t`, `i`, `t`]),
tranche(11, [`â`, `t`, `e`, `a`, `u`], [`i`, `e`, `n`, ` `, `a`]),
tranche(17, [`f`], [`s`]),
tranche(19, [`r`, `t`], [`i`, `f`])
]
\end{lstlisting}

\textbf{La structure de données \textit{différentiel}}. Un \textit{différentiel} est une liste (potentiellement vide)
de tranches \lstinline{[tr1, ..., trk]} représentant des modifications touchant des zones distinctes d’un
texte, telle que :
\begin{itemize}
\item \lstinline{début(tr1)} $<$ \lstinline{fin(tr1)} $<$ ... $<$ \lstinline{début(trk) < fin(trk)};
\item pour tout $j \in [1, k]$, pour tout $i \in$ \lstinline{[0, len(avant(trj ))-1]}, \lstinline{avant(trj )[i]} $\neq$ \lstinline{après(trj)[i]}.  
\end{itemize}

\textbf{Existence et unicité d’un différentiel par positions fixes.} Pour deux textes \lstinline{texte1} et
\lstinline{texte2} de même longueur $n$, il existe un unique différentiel \lstinline{[tr1, ..., trk]} tel que :
\begin{itemize}
\item si $k > 0$, alors $0 \leq$ \lstinline{début(tr1)} et \lstinline{fin(trk)} $\leq n $;
\item pour tout $j \in [1, k]$, \lstinline{texte1[début(trj ) : fin(trj )] = avant(trj)};
\item pour tout $j \in [1, k]$, \lstinline{texte2[début(trj ) : fin(trj )] = après(trj)};
\item pour tout $i \in [0, n-1]$, si $i \notin \cup_{1\leq j \leq k}$
\lstinline{[début(trj), fin(trj )-1]}, alors \lstinline{texte1[i] = texte2[i]}.
\end{itemize}
