%%%%%%%%%%%%%%%%%%%%%%%%%%%%%%%%%%%%%%%%%%%%%%%%
% Corrigé UPSTI
% Concours - Epreuve - Année
%%%%%%%%%%%%%%%%%%%%%%%%%%%%%%%%%%%%%%%%%%%%%%%%
% !TeX encoding = utf8
% !TeX spellcheck = fr

\documentclass[11pt]{article}

%%%%%%%%%%%%%%%%%%%%%%%%%%%%%%%%%%%%%%%%%%%%%%%%
% Package UPSTI_Document
%%%%%%%%%%%%%%%%%%%%%%%%%%%%%%%%%%%%%%%%%%%%%%%% 
\usepackage{UPSTI_Corrige_Concours}	% Squelette minimal
%\usepackage[UPSTI]{UPSTI_Corrige_Concours} % Chargement des packages UPSTI  (téléchargeables ici: https://www.upsti.fr/documents-pedagogiques/upsti-kit-de-demarrage-latex)

%---------------------------------%
% Packages personnalisés
%---------------------------------%
% Insérez ici les packages que vous utilisez habituellement





% ---

%---------------------------------%
% Paramètres du corrigé
%---------------------------------%

% ----------
% Concours
% ----------
% 0: Custom*
% 1: ATS
% 2: Banque PT
% 3: CCINP
% 4: CCP
% 5: CCS (par défaut)
% 6: E3A
% 7: ICNA
% 8: Mines AADN
% 9: Mines Ponts
% 10: X-ENS
% * Si on met la valeur 0, il faut décommenter la ligne suivante: 		
%\newcommand{\UPSTIconcoursCustom}{Concours custom}
\newcommand{\UPSTIidConcours}{5}

% ----------
% Filière
% ----------
% 0: Custom*
% 1: ATS
% 2: MP
% 3: MPI
% 4: PSI (par défaut)
% 5: PT
% 6: TSI
% 7: MP2I
% 8: MPSI
% 9: PCSI
% 10: PTSI
% * Si on met la valeur 0, il faut décommenter la ligne suivante: 		
%\newcommand{\UPSTIfiliereCustom}{Filière custom}
\newcommand{\UPSTIidFiliere}{4}

% ----------
% Epreuve
% ----------
% 0: Custom*
% 1: S2I (par défaut)
% 2: Informatique
% 3: Modélisation et informatique
% 4: Modélisation
% 5: Physique - SI
% 6: SI A
% 7: SI B
% 8: SI C
% 9: SI 1
% 10: SI 2
% * Si on met la valeur 0, il faut décommenter la ligne suivante: 		
%\newcommand{\UPSTIepreuveCustom}{Epreuve custom}
\newcommand{\UPSTIidEpreuve}{1}

% ----------
% Session
% ----------
\newcommand{\UPSTIsession}{2023}

% ----------
% Titre de l'épreuve (souvent, le nom du support)
% ----------
\newcommand{\UPSTItitreEpreuve}{Exosquelette Atalante}
% Si le nom est trop long pour l'entête, on peu décommenter la ligne suivante:
%\newcommand{\UPSTItitreEpreuveRaccourci}{Titre raccourci}      

%----------------------------------------------- 
\UPSTIprepareDocument		% "Compile" les variables
%%%%%%%%%%%%%%%%%%%%%%%%%%%%%%%%%%%%%%%%%%%%%%%% 


%%%%%%%%%%%%%%%%%%%%%%%%%%%%%%%%%%%%%%%%%%%%%%%% 
% Début du document
%%%%%%%%%%%%%%%%%%%%%%%%%%%%%%%%%%%%%%%%%%%%%%%% 
\begin{document}
\UPSTIpreambuleEpreuve	% Affichage du préambule de l'épreuve

%---------------------------------%
% DEBUT du contenu
%---------------------------------%

%\UPSTItitrePartieCorrige{Exosquelette Atalante}

\section{Mise en évidence de la problématique lors d'une marche en ligne droite}

\UPSTIobjectif{Reformuler le cahier des charges global en termes de précision de façon à l'exprimer pour chacun des axes et mettre en évidence la nécessité de la prise en compte du couplage entre les axes dans la synthèse de la loi de commande.}


\UPSTIquestion Déterminer les expressions de $Y$ et $Z$, en fonction de $\theta_{1}, \theta_{2}, L_{2}$ et $L_{3}$.
\begin{UPSTIcorrige}

\end{UPSTIcorrige}

\UPSTIquestion À l'aide du résultat de la question 1, écrit à la position $\left(Y_{0}, Z_{0}\right)$, puis à la position $\left(Y_{0}+\Delta_{Y}, Z_{0}+\Delta_{Z}\right)$, déterminer les expressions de $\Delta_{Y}$ et $\Delta_{Z}$, en fonction de $\theta_{1,0}, \theta_{2,0}, \Delta_{\theta}, L_{2}$ et $L_{3}$.
\begin{UPSTIcorrige}

\end{UPSTIcorrige}

\UPSTIquestion Déterminer alors $\Delta_{Y Z}$, la norme de la variation de positionnement total du point $D$ dans le plan $\left(\vec{y}_{1}, \vec{z}_{1}\right)$, en fonction de $\theta_{2,0}, \Delta_{\theta}, L_{2}$ et $L_{3}$.
\begin{UPSTIcorrige}

\end{UPSTIcorrige}

\UPSTIquestion À partir de la figure 5, déterminer, parmi les valeurs proposées, l'erreur maximale admissible sur les axes $S_{\theta, \max }$ de l'exosquelette Atalante afin d'éviter la chute du patient.
\begin{UPSTIcorrige}

\end{UPSTIcorrige}

\UPSTIquestion À partir de la figure 6 et en justifiant la réponse, conclure sur la capacité des asservissements réalisés sans prise en compte du couplage entre les axes à respecter l'exigence 1.2.1.1.


\section{Élaboration et analyse d'un modèle dynamique de l'exosquelette}

\UPSTIobjectif{Définir un modèle dynamique de l’exosquelette et montrer la nécessité de mettre en place un asservissement.}

\subsection{Comportement dynamique de l'exosquelette}

\UPSTIquestion Déterminer les expressions de $L_{0}$ et $\alpha$ en fonction de $l_{0}, L_{3}$ et $l_{3}$, puis calculer leurs valeurs numériques.
\begin{UPSTIcorrige}

\end{UPSTIcorrige}

\UPSTIquestion Déterminer l'expression de l'accélération du point $G_{3}$ (cf. annexe B et question 6) appartenant à l'ensemble \{pied+tibia\} 3 dans son mouvement par rapport au buste 1 , en fonction de $L_{0}, L_{2}, \theta_{1}, \theta_{2}$ et leurs dérivées temporelles.
\begin{UPSTIcorrige}

\end{UPSTIcorrige}

\UPSTIquestion Déterminer l'expression de la projection suivant $\vec{x}_{1}$ du moment dynamique en $A$ de l'ensemble $\{$ pied+tibia $\}$ 3 dans son mouvement par rapport au buste $1, \vec{\delta}_{A, 3 / 1} \cdot \vec{x}_{1}$, sous la forme :
$$
\vec{\delta}_{A, 3 / 1} \cdot \vec{x}_{1}=A_{1} \ddot{\theta}_{1}+A_{2} \ddot{\theta}_{2}+A_{3} \dot{\theta}_{1}^{2}+A_{4}\left(\dot{\theta}_{1}+\dot{\theta}_{2}\right)^{2}
$$
Préciser les expressions littérales de $A_{1}, A_{2}, A_{3}$ et $A_{4}$ en fonction des différentes caractéristiques géométriques, de masses et d'inerties de l'exosquelette.
\begin{UPSTIcorrige}

\end{UPSTIcorrige}

\UPSTIquestion Proposer une démarche permettant de déterminer l'expression de $C_{1}$, l'action mécanique exercée sur la cuisse 2 par l'actionneur correspondant. Préciser le(les) ensemble(s) isolé(s), le(s) bilan(s) des actions mécaniques extérieurs, le(s) théorème(s) utilisé(s) et la(les) équation(s) utile(s).
\begin{UPSTIcorrige}

\end{UPSTIcorrige}

\UPSTIquestion Déterminer l'expression de $C_{1}$ en fonction de $\theta_{1}, \theta_{2}$, leurs différentes dérivées, de $C_{\textrm {hanche }}$ et des différentes caractéristiques géométriques, de masses et d'inerties de l'exosquelette.
\begin{UPSTIcorrige}

\end{UPSTIcorrige}

\UPSTIquestion  Déduire des deux équations précédentes que le modèle dynamique considéré peut s'écrire sous la forme matricielle suivante :

$$
\left(\begin{array}{l}
C_{1} \\
C_{2}
\end{array}\right)=M_{1}\left(\begin{array}{c}
\ddot{\theta}_{1} \\
\ddot{\theta}_{2}
\end{array}\right)+M_{2}\left(\begin{array}{c}
\dot{\theta}_{1} \\
\dot{\theta}_{2}
\end{array}\right)+C+M_{3}\left(\begin{array}{c}
C_{\textrm {hanche }} \\
C_{\textrm {genou }}
\end{array}\right)
$$
\begin{UPSTIcorrige}

\end{UPSTIcorrige}


où $C$ est une matrice colonne et $M_{1}, M_{2}$ et $M_{3}$ sont des matrices $2 \times 2$. Donner l'expression littérale des coefficients de $C, M_{1}, M_{2}$ et $M_{3}$ par des relations non linéaires des paramètres de mouvement $\left(\theta_{1}, \theta_{2}\right)$, leurs dérivés premières et des différentes caractéristiques géométriques, de masses et d'inerties du problème.
\begin{UPSTIcorrige}

\end{UPSTIcorrige}


\subsection{Analyse du modèle dynamique}

%Q 12. 
\UPSTIquestion Déterminer les valeurs numériques des coefficients $a_{i}$ et $b_{i}$ tels que la fonction de transfert $H_{L 4}(p)=$ $\frac{\Theta_{2}(p)}{C_{2}(p)}$ s'écrive :
$$
H_{L 4}(p)=\frac{a_{0}+a_{1} p+a_{2} p^{2}}{1+b_{1} p+b_{2} p^{2}+b_{3} p^{3}+b_{4} p^{4}}.
$$
\begin{UPSTIcorrige}

\end{UPSTIcorrige}


\UPSTIquestion Au regard de cette fonction de transfert, justifier le besoin de la mise en place d'un asservissement pour l'exosquelette.
\begin{UPSTIcorrige}

\end{UPSTIcorrige}

\section{Conception et analyse de lois de commande de l’exosquelette}
\subsection{Conception de l'asservissement en couple d'un actionneur}
\UPSTIobjectif{Élaborer l'asservissement du couple généré par un actionneur, puis le valider par une analyse de son effet sur l'exosquelette.}

\subsubsection{Modèle dynamique d'un axe}

%Q 14.
\UPSTIquestion Déterminer l'expression de l'accélération du point $G_{3}$ appartenant à l'ensemble \{pied+tibia\} 3 dans son mouvement par rapport à la cuisse 2 en fonction de $L_{0}, \theta_{2}$ et ses dérivées.
\begin{UPSTIcorrige}

\end{UPSTIcorrige}

%Q 15.
\UPSTIquestion Par application du théorème du moment dynamique à l'ensemble \{pied+tibia 3 au point B, projeté suivant la direction $\vec{x}_{1}$, donner l'expression de $C_{2}$ sous la forme :

$$
C_{2}=A_{\mathrm{eq}} \ddot{\theta}_{2}+C_{r}
$$

Préciser les expressions de $A_{\textrm{eq}}$ en fonction de $I_{x 3}, m_{3}, L_{0}$ et de $C_{r}$ en fonction de $C_{\textrm{genou}}, m_{3}, L_{0}, \alpha$ et $\theta_{2}$. Faire l'application numérique pour $A_{e q}$.
\begin{UPSTIcorrige}

\end{UPSTIcorrige}


\subsubsection{Élaboration de la commande en couple de l'actionneur}

%Q 16
\UPSTIquestion  Calculer, en donnant les expressions littérales de $K_{c}, T_{c}$, $\xi_{c}$ et $\omega_{c}$, l'expression de la fonction de transfert $H_{C}(p)=\frac{C_{m}(p)}{C_{r e f}(p)}$ sous la forme :
$$
H_{C}(p)=K_{c} \frac{1+T_{c} p}{1+2 \xi_{c} \frac{p}{\omega_{c}}+\left(\frac{p}{\omega_{c}}\right)^{2}}
$$
\begin{UPSTIcorrige}

\end{UPSTIcorrige}

%Q 17
\UPSTIquestion Déterminer l'expression littérale, puis la valeur numérique, de $K$ afin de respecter le cahier des charges en termes de rapidité (rappel : $t_{r, 5 \%} \omega_{c} \approx 5$ pour un régime apériodique critique).
\begin{UPSTIcorrige}

\end{UPSTIcorrige}

%Q 18
\UPSTIquestion Déterminer l'expression littérale, puis la valeur numérique, de $T$ afin de respecter le cahier des charges en termes de stabilité.
\begin{UPSTIcorrige}

\end{UPSTIcorrige}

%Q 19
\UPSTIquestion  Déterminer l'expression littérale, puis la valeur numérique, de $K_{1}$ afin d'obtenir $K_{C}=r$ pour $H_{C}(p)$. Justifier la volonté d'obtenir cette valeur de gain pour $H_{C}(p)$.
\begin{UPSTIcorrige}

\end{UPSTIcorrige}

%Q 20
\UPSTIquestion À partir de la figure 10, discuter du respect des exigences de l'asservissement en couple.
\begin{UPSTIcorrige}

\end{UPSTIcorrige}

\subsubsection{Vérification de la cohérence de la commande en couple sur l’exosquelette}

%Q 21. 
\UPSTIquestion À partir des figures 11 et 12, conclure sur la possibilité de considérer l'asservissement en couple comme amenant une action transparente pour le patient.
\begin{UPSTIcorrige}

\end{UPSTIcorrige}

\subsection{Synthèse de la loi de commande de l'exosquelette}

\UPSTIobjectif{Élaborer une loi de commande en phase de rééducation à la proprioception de la verticalité et de la marche, et en phase de renforcement musculaire.}
\begin{UPSTIcorrige}

\end{UPSTIcorrige}

%Q 22. 
\UPSTIquestion Préciser l'expression de la matrice $M$ introduite précédemment en fonction de $M_{1}$ et $M_{3}$.
\begin{UPSTIcorrige}

\end{UPSTIcorrige}



\end{document}
