\newpage

\section*{ANNEXE 1 - Paramétrage mécanique \label{2023_MP_CCINP_ann_01}}
\subsection*{Paramètres généraux :}
Soient :

\begin{itemize}
  \item $\mathbf{0}$ le sol, $\mathbf{S 1}$ le châssis de la foreuse, $\mathbf{S 2}$ sa tourelle et son mât et $\mathbf{S 3}$ l'ensemble \{table de forage + outil\} ;
  \item $\rep{0}=(O ; \vec{x}, \vec{y}, \vec{z})$ le repère attaché aux solides $\mathbf{S 0}$ et $\mathbf{S 1}$;
  \item $\mathcal{B}_{2}=\left(\vec{x}_{2}, \vec{y}_{2}, \vec{z}\right)$ la base attachée aux solides $\mathbf{S 2}$ et $\mathbf{S} \mathbf{3}$ telle que $\left(\vec{x}, \vec{x}_{2}\right)=\theta$ où $\theta$ est connu ;
  \item $\boldsymbol{\Sigma}=\{\mathbf{S} \mathbf{1}, \mathbf{S 2}, \mathbf{S 3}\}$ l'ensemble de la foreuse, de centre de gravité $G$ tel que $\overrightarrow{\mathrm{OG}}=r \vec{x}_{2}+\boldsymbol{z}_{G} \vec{z}$;
  \item $M=186,5$ tonnes la masse de l'ensemble $\boldsymbol{\Sigma}$ et $m=18$ tonnes la masse de $\mathbf{S 3}$ seul ;
  \item $2 F_{w} \vec{z}$, connu, l'effort du câble d'avance sur $\mathbf{S 3}$. La masse du câble est négligée dans la suite ;
  \item $F_{\text {sol }} \vec{z}$, inconnu, l'effort de forage du sol $\mathbf{0}$ sur l'outil de forage $\mathbf{S} 3$ au point F , connu, défini par $\overrightarrow{\mathrm{OF}}=R \vec{x}_{2} ;$
  \item $-g \vec{z}$ où $g=\SI{9,8}{m .s^{-2}}$, l'accélération de la pesanteur terrestre.
\end{itemize}

%\begin{figure}[h]
%\begin{center}
%  \includegraphics[width=\textwidth]{2025_10_26_292d8f726e32404b86d3g-16(1)}
%\captionsetup{labelformat=empty}
%\caption{Système réel}
%\end{center}
%\end{figure}

%\begin{figure}[h]
\begin{center}
%\captionsetup{labelformat=empty}
%\caption{Modèle}
  \includegraphics[width=\textwidth]{ann_01}
\end{center}
%\end{figure}

\subsubsection*{Paramétrage 1 : modèle avec efforts ponctuels entre le sol et la foreuse}
\begin{itemize}
  \item $F_{g} \vect{z}$, inconnu, l'effort du sol $\mathbf{0}$ sur $\mathbf{S 1}$, supposé ponctuel au centre $I$ de la surface de contact entre la chenille gauche $c g$ et le sol tel que $\|\overrightarrow{OI}\|=a=\SI{2,1}{m}$;
  \item $F_{d} \vec{z}$, inconnu, l'effort du sol $\mathbf{0}$ sur $\mathbf{S 1}$, supposé ponctuel au centre $J$ de la surface de contact entre la chenille droite $c d$ et le sol tel que $\|\overrightarrow{OJ}\|=a=\SI{2,1}{m}$.
\end{itemize}

\subsubsection*{Paramétrage 2 : modèle avec répartition de pression entre le sol et la foreuse}
On note :

\begin{itemize}
  \item $\mathrm{P}(x, y, 0)$, un point courant de contact entre le sol et les chenilles. Attention, $x$ est négatif sur la figure ci-dessous. Les grandeurs $d x$ et $d y$ sont les dimensions du domaine surfacique élémentaire autour du point P entre le sol et les chenilles;
  \item $p_{g}(y)=A \cdot \dfrac{y}{L}+B$, la pression du sol 0 sur la chenille gauche $c g$ au point $\mathrm{P}(x, y, 0)$ où $A$ et $B$, homogènes à des pressions, sont inconnues et à déterminer ;
  \item $p_{d}(y)=C \cdot \dfrac{y}{L}+D$, la pression du sol 0 sur la chenille droite $c d$ au point $\mathrm{P}(x, y, 0)$ où $C$ et $D$, homogènes à des pressions, sont inconnues et à déterminer ;
  \item $L=\SI{5,4}{m}$, la longueur et $I=\SI{1}{m}$ la largeur de chaque chenille ;
  \item $a=\SI{2,1}{m}$, la distance moyenne sur l'axe $\vec{x}$ d'une chenille au centre O de la machine.\\
\end{itemize}
\begin{center}
\includegraphics[width=.8\textwidth]{2025_10_26_292d8f726e32404b86d3g-17}
\end{center}

\section*{ANNEXE 2 - Rappels des syntaxes en Python}
Pour ce sujet, on admet que les imports suivants ont été réalisés en amont :\\
from numpy import *\\
import matplotlib.pyplot as plt\\
Le tableau suivant récapitule quelques éléments de syntaxe utiles :

\begin{center}
\begin{tabular}{|l|l|}
\hline
 & Syntaxe Python \\
\hline
Définir un tableau à une dimension & \( \begin{aligned} & \mathrm{L}=[1,2,3] \text { (liste) } \\ & \mathrm{V}=\text { array }([1,2,3]) \text { (vecteur) } \end{aligned} \) \\
\hline
Accéder à un élément & v[0] renvoie 1 (L[0] également) \\
\hline
Créer une nouvelle figure & plt.figure() \\
\hline
Tracer une courbe & $\mathrm{plt} . \mathrm{plot}(\mathrm{X}, \mathrm{Y})$ où X et Y , de même taille, sont respectivement la liste des abscisses et des ordonnées de la courbe. \\
\hline
\end{tabular}
\end{center}

\section*{ANNEXE 3 - Exemple d'un rapport de chantier}
\begin{center}
\includegraphics[width=\textwidth]{2025_10_26_292d8f726e32404b86d3g-18}
\end{center}
