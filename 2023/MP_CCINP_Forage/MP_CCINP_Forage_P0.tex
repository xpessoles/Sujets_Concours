\section*{Contrôle d'une machine de forage }
\subsection*{Présentation générale \label{2023_MP_CCINP_sec_00}}
Dans le domaine du génie civil, toute structure architecturale a besoin d'une fondation faisant office de liaison entre celle-ci et le sol. Elle permet d'assurer la transmission des charges et leur répartition dans le sol. Lorsque le sol résistant se trouve à une très grande profondeur, ou dans le cas d'une structure très importante, il est nécessaire de créer une fondation profonde composée de pieux en béton armé. L'excavation des terres se réalise alors grâce à un engin appelé foreuse (ou machine de forage).

Comme l'illustre la figure \ref{2023_MP_CCINP_fig_01}, ces pieux de fondation profonde sont réalisés en plusieurs étapes :

\begin{itemize}
  \item étape 1 : positionnement de la foreuse et de l'outil au-dessus du point d'implantation du pieu ;
  \item étape 2: forage profond jusqu'au sol résistant ;
  \item étape 3: mise en place de l'armature (treillis de fer) du pieu ;
  \item étape 4 : coulée du béton ;
  \item étape 5 : retrait de la machine et séchage du pieu.
\end{itemize}

\begin{figure}[h]
\begin{center}
  \includegraphics[width=.8\textwidth]{2025_10_26_292d8f726e32404b86d3g-02}
%\captionsetup{labelformat=empty}
%Figure 1 -
\caption{Réalisation d'un pieu de forage profond\label{2023_MP_CCINP_fig_01}}
\end{center}
\end{figure}

Une machine de forage est un système riche en sous-systèmes. Une description fonctionnelle partielle est donnée figure \ref{2023_MP_CCINP_fig_02} et une description structurelle, simplifiée, se situe en figure \ref{2023_MP_CCINP_fig_03}.

\begin{figure}[h]
\begin{center}
  \includegraphics[width=\textwidth]{2025_10_26_292d8f726e32404b86d3g-02(1)}
%\captionsetup{labelformat=empty}Figure 2 - 
\caption{Diagramme des exigences partiel \label{2023_MP_CCINP_fig_02}}
\end{center}
\end{figure}


Une tourelle, comportant la cabine de pilotage, les différents moteurs de la foreuse et des contrepoids à l'arrière, est montée sur un châssis équipé de chenilles. La tourelle est orientable autour d'un axe vertical à $360^{\circ}$ par rapport au châssis et porte un mât de levage. Ce mât, dont la portée et la verticalité par rapport à la tourelle sont réglables à l'aide de vérins, guide une table de forage en translation grâce à un câble d'avance lui-même actionné par un treuil d'avance. En phase de forage, pendant que la table de forage descend, celle-ci actionne en rotation la tige Kelly au bout de laquelle est montée l'outil, généralement une tarière (vis sans fin). Le mouvement d'avance combiné au mouvement de rotation de l'outil permet le forage. De plus, on note que la table de forage est équipée d'un système de suspension permettant d'amortir, pour le reste de la machine, les vibrations dues aux efforts de forage.

Dans ce sujet, il ne sera étudié que la stabilité globale de la machine (Partie \ref{2023_MP_CCINP_sec_01}) et l'axe d'avance (Partie \ref{2023_MP_CCINP_sec_02}). Le contrôle de ces fonctionnalités est soumis aux exigences présentes dans le diagramme des exigences partiel de la figure \ref{2023_MP_CCINP_fig_02}. Enfin, une dernière partie explorera une manière d'éditer un rapport de chantier (Partie \ref{2023_MP_CCINP_sec_03}). Toutes ces parties sont indépendantes.

\begin{figure}[h]
\begin{center}
  \includegraphics[width=.45\textwidth]{2025_10_26_292d8f726e32404b86d3g-03(1)}
%\captionsetup{labelformat=empty}Figure 3 - 
\caption{Principaux constituants d'une foreuse \label{2023_MP_CCINP_fig_03}}
\end{center}
\end{figure}


