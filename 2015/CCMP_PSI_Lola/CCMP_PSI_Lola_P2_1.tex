\section{Contrôle de la posture de LOLA}%Partie 2 : 
\label{ccmp_psi_2015:p02}
Pour assurer une marche rapide et stable de LOLA, la méthode choisie est le contrôle de la verticalité du tronc du robot (figure 5). Le haut du corps (tronc, bras, tête) sera maintenu vertical en réalisant un asservissement de position angulaire au niveau de l'articulation de la hanche. L'action mécanique de redressement est développée par l'ensemble de motorisation de tangage autour de l'axe ( $\mathrm{O}_{\mathrm{T}}, \overrightarrow{\mathrm{x}}_{0}$ ). Les performances à vérifier dans cette partie sont définies par les exigences suivantes :

\begin{figure}[h]
\begin{center}
  \includegraphics[width=\textwidth]{2025_10_26_7c3d1b71880592bd5f63g-05}
\captionsetup{labelformat=empty}
\caption{Figure 5 \label{ccmp_psi_2015:fig05}}
\end{center}
\end{figure}

La chaîne structurelle permettant de modifier la posture du haut du corps autour de l'axe de tangage est représentée sur la figure 6 (page 6 ). Elle est composée d'un moteur électrique $(1,2)$ synchrone à aimants\\
permanents piloté par un variateur électronique, d'un réducteur Harmonic-Drive ${ }^{©}$ (3) de rapport de réduction $1 / 100$, d'un codeur incrémental (5) ainsi que d'un codeur angulaire absolu (6+7).

Une centrale inertielle équipée d'un accéléromètre, d'un gyroscope et d'une unité de traitement permet d'obtenir en temps réel la valeur de l'angle d'inclinaison du haut du corps par rapport à la verticale.

L'objectif de cette partie est de mettre en place un modèle du maintien vertical du tronc de LOLA et de déterminer une structure de commande permettant d'assurer les performances du cahier des charges de l'exigence 1.3.2.

\begin{figure}[h]
\begin{center}
  \includegraphics[width=\textwidth]{2025_10_26_7c3d1b71880592bd5f63g-06}
\captionsetup{labelformat=empty}
\caption{Figure \label{ccmp_psi_2015:fig06}}
\end{center}
\end{figure}

Les performances dynamiques de l'axe de tangage doivent vérifier les critères suivants:

\begin{center}
\begin{tabular}{|l|l|l|}
\hline
\multicolumn{3}{|c|}{Sous-exigence 1.3.2.d: la performance dynamique de chaque axe permet de modifier la posture} \\
\hline
Critère & Niveau & Flexibilité \\
\hline
Marge de phase & $\mathrm{M} \varphi=50^{\circ}$ & Mini \\
\hline
Erreur statique & $0^{\circ}$ & [ $-0.5^{\circ} ;+0.5^{\circ}$ ] \\
\hline
Bande passante à 0 dB en boucle ouverte & $\omega_{\text {BP }}=50 \mathrm{rad} . s^{-1}$ & Mini \\
\hline
Temps de réponse à 5\% & $0,2 \mathrm{~s}$ & Maxi \\
\hline
Dépassement & $1^{\circ}$ & Maxi \\
\hline
\end{tabular}
\end{center}


\subsection{Modèle de connaissance de la dynamique de tangage \label{ccmp_psi_2015:p21}}
Le modèle mécanique utilisé pour mener notre étude est donné sur la figure 7. L'association des liaisons entre le tronc et les jambes au niveau de la hanche est équivalente, dans le plan sagittal ( $\mathrm{O}_{\mathrm{T}}, \overrightarrow{\mathrm{y}}_{0,}, \overrightarrow{\mathrm{z}}_{0}$ ), à une liaison pivot d'axe $\left(\mathrm{O}_{\mathrm{T}}, \overrightarrow{\mathrm{x}}_{0}\right)$. Le tronc sera considéré comme un solide admettant le plan ( $\mathrm{O}_{\mathrm{T}}, \overrightarrow{\mathrm{y}}_{0,}, \overrightarrow{\mathrm{z}}_{0}$ ) comme plan de symétrie. Le cahier des charges stipule que LOLA doit pouvoir marcher à la vitesse de $5 \mathrm{~km} / \mathrm{h}$. Cette vitesse est atteinte en 1 s lors de la première foulée. La loi de commande en vitesse correspondante est représentée

\begin{figure}[h]
\begin{center}
  \includegraphics[width=\textwidth]{2025_10_26_7c3d1b71880592bd5f63g-06(1)}
\captionsetup{labelformat=empty}
\caption{Figure 7\label{ccmp_psi_2015:fig07}}
\end{center}
\end{figure}

\begin{figure}[h]
\begin{center}
  \includegraphics[width=\textwidth]{2025_10_26_7c3d1b71880592bd5f63g-06(2)}
\captionsetup{labelformat=empty}
\caption{Figure 7 \label{ccmp_psi_2015:fig07bis}}
\end{center}
\end{figure}

Le mouvement de marche est imposé et modélisé par le torseur cinématique en $\mathrm{O}_{\mathrm{T}}$ du mouvement du tronc 1 par rapport au sol 0 :

$$
\left\{V_{1 / 0}\right\}=\left\{\begin{array}{c}
\dfrac{\d \alpha}{\d t} \vec{x}_{0} \\
v(t) \vec{y}_{0}
\end{array}\right\}_{\alpha_{\mathrm{T}}}
$$

Les caractéristiques d'inertie du tronc 1 de LOLA sont :

\begin{itemize}
  \item la matrice d'inertie en $\mathrm{O}_{\mathrm{T}}: \mathrm{I}\left(\mathrm{O}_{\mathrm{T}}, 1\right)=\left[\begin{array}{ccc}\mathrm{A}_{1} & 0 & 0 \\ 0 & \mathrm{~B}_{1} & -\mathrm{D}_{1} \\ 0 & -\mathrm{D}_{1} & \mathrm{C}_{1}\end{array}\right]_{\mathrm{B}_{1}}$
  \item position du centre de gravité : $\overline{\mathrm{O}_{\mathrm{T}} \overrightarrow{\mathrm{G}}_{\mathrm{T}}}=\mathrm{Z}_{\mathrm{G}} \cdot \vec{Z}_{1}$
  \item masse: $\mathrm{m}_{1}$
  \item l'accélération de la pesanteur sera prise égale à $\mathrm{g}=9,81 \mathrm{~m} \cdot \mathrm{~s}^{-2}$.
\end{itemize}

L'axe de sortie du réducteur exerce un couple de redressement sur le tronc 1 modélisé par le torseur couple suivant : $\left\{T_{h d \rightarrow 1}\right\}=\left\{\begin{array}{c}\overrightarrow{0} \\ C_{R} \vec{x}_{0}\end{array}\right\}_{O_{T}}$. L'angle a sera supposé faible pendant le mouvement : ainsi $\cos \alpha \sim 1$ et $\sin \alpha \sim \alpha$

%\section*{Question 7 :}
\question{\label{ccmp_psi_2015:q07}
Proposer une démarche de résolution afin d'obtenir l'équation différentielle du mouvement reliant $\alpha$ et ses dérivées successives aux données du problème. Effectuer un bilan des actions mécaniques extérieures au système matériel isolé.}

%\section*{Question 8 :}
\question{\label{ccmp_psi_2015:q08}Développer l'ensemble des calculs pour déterminer l'équation différentielle reliant $\alpha$ et ses dérivées successives aux données du problème.}

Le contrôle de l'angle s'effectue par l'intermédiaire du moteur asservi en position, suivi du réducteur HarmonicDrive ${ }^{®}$ de rapport de réduction $\mathrm{r}=\frac{1}{100}$. Le moment d'inertie de l'arbre moteur suivant son axe de rotation est noté $\mathrm{J}_{\mathrm{m}}$, le couple moteur exercé sur l'arbre d'entrée du réducteur est noté $\mathrm{C}_{\mathrm{m}}$. Le réducteur HarmonicDrive ${ }^{®}$ sera considéré sans masse. La masse de l'arbre moteur est négligeable devant l'ensemble des autres grandeurs inertielles. Une étude dynamique a permis de montrer que : $C_{R}=\frac{C_{m}}{r}-\frac{J_{m}}{r^{2}} \cdot \frac{d^{2} \alpha(t)}{d t^{2}}$. Ainsi, l'équation différentielle du mouvement devient :


\begin{equation*}
J_{e q} \frac{d^{2} \alpha(t)}{d t^{2}}-m_{1} g Z_{G} \alpha(t)=m_{1} Z_{G} \frac{d v(t)}{d t}+\frac{C_{m}(t)}{r} \tag{1}
\end{equation*}


$\mathrm{J}_{\text {eq }}$ est le moment d'inertie équivalent de l'ensemble du tronc ramené sur l'axe moteur.

