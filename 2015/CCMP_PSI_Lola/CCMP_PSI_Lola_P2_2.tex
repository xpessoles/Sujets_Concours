%2. 
\subsection{Modèle du contrôle actif de la position verticale. \label{ccmp_psi_2015:p22}}
On note $\Gamma(t)=\dfrac{\d v(t)}{\d t}$. Les conditions de Heaviside sont vérifiées. Le schéma-bloc du contrôle de la position angulaire du tronc de LOLA est représenté sur l'annexe 2. La consigne angulaire est nulle afin de garder le tronc du robot vertical: $\alpha_{c}(t)=0$. Les transformées de Laplace des fonctions seront notées en majuscules et le paramètre de Laplace sera noté $p$.\\
Le comportement du moteur sera considéré comme celui d'un moteur à courant continu dont les équations de comportement sont les suivantes: $u_{c}(t)=e(t)+L \dfrac{\d i(t)}{\d t}+R i(t) ; e(t)=k_{e} \omega_{m}(t)$ et $C_{m}(t)=k_{c} i(t)$.

%\section*{Question 9 :}
\question{\label{ccmp_psi_2015:q09}Indiquer sur le document-réponse les fonctions de transfert des blocs $B_{1}, B_{2}, B_{3}, B_{4}, B_{5}, B_{6}$ et $B_{7}$ ainsi que l'expression de la fonction de transfert $H_{1}(p)$.}

Afin d'analyser la stabilité de cet asservissement, nous cherchons à déterminer la fonction de transfert en boucle ouverte du système non-corrigé : $F(p)=\dfrac{\alpha(p)}{U_{c}(p)}$ en supposant la perturbation nulle.

%\section*{Question 10 :}
\question{\label{ccmp_psi_2015:q10}
Déterminer la fonction de transfert de la boucle dynamique $H_{\text {dyn }}(p)=\dfrac{\alpha(p)}{C_{m}(p)}$ en supposant la perturbation nulle.}

%\section*{Question 11 :}
\question{\label{ccmp_psi_2015:q11}Déterminer la fonction de transfert en boucle ouverte non-corrigée de l'asservissement $F(p)=\dfrac{\alpha(p)}{U_{c}(p)}$. Indiquer son ordre, sa classe et donner son gain statique $K$ en fonction des données.}

Une simulation numérique permet de montrer que $F(p)$ est de la forme $\dfrac{K}{\left(1+\tau_{1} p\right)\left(-1+\tau_{1} p\right)\left(1+\tau_{2} p\right)}$. Les diagrammes de Bode de cette fonction de transfert sont donnés sur le document-réponse.

%\section*{Question 12 :}
\question{\label{ccmp_psi_2015:q12}En analysant les diagrammes de Bode du document-réponse, déterminer les valeurs de $\tau_{1}, \tau_{2}$ et K . Justifier en complétant les diagrammes du document-réponse avec les diagrammes asymptotiques de gain et de phase.}

Pour la suite de l'étude, nous simplifierons $F(p)$ sous la forme suivante : $\dfrac{K}{\left(1+\tau_{1} p\right)\left(-1+\tau_{1} p\right)}$.

%\section*{Question 13 :}
\question{\label{ccmp_psi_2015:q13}Justifier le choix de cette simplification.}

%\section*{Question 14 :}
\question{\label{ccmp_psi_2015:q14}Expliquer pourquoi le critère du revers ne peut pas être appliqué pour étudier la stabilité en boucle fermée.}

Afin de résoudre ce problème, il est décidé d'asservir la chaîne directe en position et en vitesse. Pour cela, la centrale inertielle permet de mesurer l'angle de tangage $\alpha(t)$ ainsi que la vitesse angulaire $\dfrac{\d \alpha(t)}{\d t}$. L'asservissement ainsi réalisé est présenté sous la forme du schéma-bloc de la figure \ref{ccmp_psi_2015:fig08} (page 9).\\
$U_{c}(p)$ est la tension de commande en sortie du correcteur. La fonction de transfert de la centrale inertielle sera prise égale à $\mathrm{H}_{\mathrm{ci}}(\mathrm{p})=\mathrm{K}_{1}(\mathrm{p}+1)$.

\begin{figure}[h]
\begin{center}
  \includegraphics[width=.6\textwidth]{2025_10_26_7c3d1b71880592bd5f63g-09}
%\captionsetup{labelformat=empty}
\caption{Figure 8 \label{ccmp_psi_2015:fig08}}
\end{center}
\end{figure}

%\section*{Question 15 :}
\question{\label{ccmp_psi_2015:q15}Déterminer deux conditions sur $\mathrm{K}_{1}$ pour que la fonction de transfert en boucle ouverte non-corrigée $\dfrac{\alpha(p)}{U_{c}(p)}$ soit stable. En déduire la valeur minimale de $K_{1}$.}

%\section*{Question 16 :}
\question{\label{ccmp_psi_2015:q16}Déterminer $\mathrm{K}_{1}$ pour que la fonction de transfert $\dfrac{\alpha(\mathrm{p})}{\mathrm{U}_{\mathrm{C}}(\mathrm{p})}$ ait un facteur d'amortissement $\xi=1,7$. Vérifier que cette valeur est compatible avec les conditions obtenues précédemment. En déduire les valeurs de la pulsation propre $\omega_{0}$ et du gain statique de la boucle ouverte $\mathrm{K}_{\mathrm{B} 0}$.}

Quels que soient les résultats trouvés précédemment, nous utiliserons les expressions suivantes pour la suite de l'étude: $\dfrac{\alpha(p)}{U_{C}(p)}=\dfrac{K_{B O}}{1+\dfrac{2 . \xi}{\omega_{0}} \cdot p+\dfrac{p^{2}}{\omega_{0}^{2}}}$ avec $K_{B O}=1,1.10^{-3}, \xi=1,7$ et $\omega_{0}=3$ rad. $s^{-1}$. Pour répondre au cahier des charges, il est décidé d'implanter un correcteur de fonction de transfert suivante : $H_{\text {cor }}(p)=K_{p .} \dfrac{1+a T_{d} p}{1+T_{d} p}$ avec $\mathrm{a}>1$.

%\section*{Question 17 :}
\question{\label{ccmp_psi_2015:q17}Nommer ce correcteur.}

Les diagrammes de Bode de gain et de phase (pour $\mathrm{K}_{\mathrm{p}}=1$ ) de ce correcteur sont fournis en annexe 3. Afin d'assurer un gain significatif de phase, nous décidons de placer $\omega_{c}$ en $\omega_{B P}=50$ rad.s $^{-1}$, définissant ainsi la bande passante.

%\section*{Question 18 :}
\question{\label{ccmp_psi_2015:q18}Déterminer la valeur du paramètre a pour que le correcteur permette d'assurer la marge de phase du cahier des charges. En déduire la valeur de $T_{d}$.}

%\section*{Question 19 :}
\question{\label{ccmp_psi_2015:q19}
Déterminer le gain $\mathrm{K}_{\mathrm{p}}$ pour que le critère de bande passante du cahier des charges soit bien vérifié.}

La stabilité du tronc étant assurée, nous devons maintenant analyser les performances en précision et rapidité de l'asservissement de position angulaire. La consigne est nulle, ainsi seule la perturbation va écarter le tronc du robot de sa posture verticale. Cette perturbation provient du mouvement de marche souhaité c'est à dire de l'accélération subie $\Gamma(t)=\dfrac{\d v(t)}{\d t}$. Avec les réglages du correcteur, une simulation numérique a permis de tracer la réponse temporelle du système pour une perturbation $\Gamma(t)$ respectant la loi de vitesse représentée sur la figure 7 de la page 6. Cette réponse est tracée sur l'annexe 4.

%\section*{Question 20 :}
\question{\label{ccmp_psi_2015:q20}
Justifier l'allure de la réponse temporelle. Déterminer graphiquement sur le document réponse le temps de réponse à $5 \%$, le dépassement maximal et l'erreur statique. Conclure sur la capacité du correcteur à vérifier l'ensemble des critères du cahier des charges.}

